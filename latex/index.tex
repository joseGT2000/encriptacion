Desde la edad antigua se han encriptado mensajes diplomáticos, militares o administrativos para conseguir comunicaciones seguras, a salvo de espías. En la actualidad, la Criptografía desempeña un papel de gran importancia dentro de la Informática. Su uso es prácticamente cotidiano, ya que elementos tan familiares como los passwords, los números de tarjetas, etc., se guardan encriptados e incluso hay programas que permiten cifrar y descifrar los mensajes de correo electrónico y, en general, cualquier documento en soporte digital. Un mensaje puede encriptarse por permutación o por sustitución de sus caracteres, siguiendo unas reglas predefinidas y quizá introduciendo algún elemento adicional (claves, números aleatorios, ...). Obviamente, si se desea descifrar un mensaje ya encriptado, resulta esencial que las reglas de cifrado sean reversibles. En esta práctica proponemos dos métodos de encriptación, uno de cada tipo. Para cada caso solo mostramos el algoritmo de encriptación, el de desencriptación es fácilmente deducible. 